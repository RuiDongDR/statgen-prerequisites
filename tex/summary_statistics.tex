% Generated slides for Genotype Coding
% Include this file in your main Beamer presentation

\begin{frame}{Section}
\centering
\Huge{Summary Statistics}
\end{frame}


\begin{frame}{Intuitional Description}
Summary statistics in genetics capture \textbf{the association between genetic variants and traits}, allowing researchers to share and analyze meaningful associations \textbf{without requiring access to the complete individual-level genotype and phenotype data}.
\end{frame}

\begin{frame}{Graphical Summary}
\includesvg[width=0.8\textwidth]{./cartoons/summary_statistics.svg}
\end{frame}


\begin{frame}{Key Formula}
$$
Z = \frac{\beta}{\text{SE}(\beta)}
$$

This Z-score formula represents the most fundamental calculation in statistical genetics summary statistics. It standardizes the effect size ($\beta$) of a genetic variant by its standard error ($\text{SE}$), creating a normalized measure of association strength. The importance of this formula lies in its ability to quantify statistical significance while accounting for estimation precision. This single value determines p-values, enables cross-variant comparison, and forms the foundation for downstream analyses including meta-analysis, polygenic risk scoring, and genetic correlation estimation.

\end{frame}


\begin{frame}{Technical Details: Single Marker Linear Regression Model}
Assuming that we use the single marker linear regression model (and everything is applicable to logistic regression or multiple markers models etc.), then for a trait vector $ \mathbf{y} $ and a genetic variant $ \mathbf{X}_j $ across $ N $ individuals:

$$
\mathbf{Y} = \mathbf{X}_j \beta_j + \boldsymbol{\epsilon}, \quad \boldsymbol{\epsilon} \sim N(0, \sigma^2)
$$

Where:
\begin{itemize}
\item $ \mathbf{Y} $ is the $ N \times 1 $ vector of trait values for $ N $ individuals.
\item $ \mathbf{X}_j $ is the $ N \times 1 $ vector of genotypes for variant $ j $ across all individuals.
\item $ \beta_j $ is the effect size of the genetic variant $ j $.
\item $ \boldsymbol{\epsilon} $ is the $ N \times 1 $ vector of error terms (residuals), assumed to follow a normal distribution with mean 0 and variance $ \sigma^2 $.
\end{itemize}

\end{frame}

\begin{frame}{Technical Details: From OLS to Summary Staistics}

And using \textbf{Ordinary Least Squares (OLS)} method (mentioned in Section [FIXME -- refer to the section label]), which minimizes the residual sum of squares (RSS) between the observed and predicted trait values, and both $\mathbf{x}$ and $\mathbf{Y}$ are scaled to have mean 0 and variance 1, we get the following formula for the OLS estimator of $ \beta_j$:

$$
\hat{\beta}_j = \frac{\mathbf{X}_j^T \mathbf{Y}}{\mathbf{X}_j^T \mathbf{X}_j} = \mathbf{X}_j^T \mathbf{Y} = \text{Cov}(\mathbf{X}_j, \mathbf{Y}) = r_{\mathbf{X_j}, \mathbf{Y}} 
$$

And the variance of $ \hat{\beta}_j $ can be calculated as:

$$
\text{Var}(\hat{\beta}_j) = \frac{\sigma^2}{\mathbf{X}_j^T \mathbf{X}_j} = \sigma^2
$$

Where $ \sigma^2 $ is the residual variance, which is typically estimated from the residuals of the model.

\end{frame}

\begin{frame}{Technical Details: Z-score and P-value}

\begin{itemize}
\item Z-score 
\begin{itemize}
\item Z-score is the standardized effect size
\end{itemize}
\end{itemize}

    $$
    Z_j = \frac{\hat{\beta}_j}{SE(\hat{\beta}_j)} = \frac{\hat{\beta}_j}{\sqrt{\text{Var}(\hat{\beta}_j)}} = \frac{\hat{\beta}_j}{\sigma}
    $$

\begin{itemize}
\item Under the null hypothesis of no association ($\beta_j = 0$), $Z_j$ follows a standard normal distribution, which forms the basis for statistical testing.
\end{itemize}

\begin{itemize}
\item P-value
\begin{itemize}
\item Two-sided p-value: 
\end{itemize}
\end{itemize}
    $$
    P = 2 \times (1 - \Phi(|Z|))
    $$
\begin{itemize}
\item $\Phi$ is the cumulative distribution function of the standard normal distribution
\item Represents the probability of observing an effect as extreme as β by chance
\end{itemize}

\end{frame}

\begin{frame}{Technical Details: Example GWAS Summary Statistics Table -- part A}

\begin{table}[h]
\centering
\resizebox{\textwidth}{!}{
\begin{tabular}{|c|c|c|c|c|c|c|c|c|c|c|}
\hline
SNP (rsID) & CHR & BP & A1 & A2 & MAF & BETA & SE & Z-score & P-value & N \\ \hline
rs12345 & 1 & 10583 & A & G & 0.12 & 0.045 & 0.010 & 4.50 & 1.2e-06 & 100000 \\ \hline
rs67890 & 2 & 20345 & C & T & 0.35 & -0.030 & 0.008 & -3.75 & 5.4e-04 & 95000 \\ \hline
rs54321 & 18 & 45678 & G & A & 0.22 & 0.060 & 0.012 & 5.00 & 2.1e-07 & 102000 \\ \hline
\end{tabular}}
\end{table}

\begin{itemize}
\item \textbf{SNP (rsID)}: Identifier for the single nucleotide polymorphism, such as \texttt{rs12345} or \texttt{chr21:295472:G:A}.  
\item \textbf{CHR}: Chromosome number where the SNP is located. Other common names: \texttt{chrom}, \texttt{chromosome}  
\item \textbf{BP}: Genomic position of the SNP, specific to a reference genome assembly (e.g., GRCh37 or GRCh38). Other common names: \texttt{pos}, \texttt{position}  
\end{itemize}

\end{frame}

\begin{frame}{Technical Details: Example GWAS Summary Statistics Table -- part B}

\begin{table}[h]
\centering
\resizebox{\textwidth}{!}{
\begin{tabular}{|c|c|c|c|c|c|c|c|c|c|c|}
\hline
SNP (rsID) & CHR & BP & A1 & A2 & MAF & BETA & SE & Z-score & P-value & N \\ \hline
rs12345 & 1 & 10583 & A & G & 0.12 & 0.045 & 0.010 & 4.50 & 1.2e-06 & 100000 \\ \hline
rs67890 & 2 & 20345 & C & T & 0.35 & -0.030 & 0.008 & -3.75 & 5.4e-04 & 95000 \\ \hline
rs54321 & 18 & 45678 & G & A & 0.22 & 0.060 & 0.012 & 5.00 & 2.1e-07 & 102000 \\ \hline
\end{tabular}}
\end{table}

\begin{itemize}
\item \textbf{A1 and A2}: The two alleles at the variant site, with one designated as the effect allele. Other common names: \texttt{REF} and \texttt{ALT}, \texttt{Effect allele} and \texttt{Other allele}
\begin{itemize}
\item Note that one allele may be effect allele in one study but alternative allele in another.
\end{itemize}
\item \textbf{MAF}: The frequency of the less common allele for variant $j$ in the sample.
\item \textbf{EAF}: The frequency of the effect allele for variant $j$ in the sample (note that this doesn't have to be the minor allele) (sometimes called \texttt{RAF}: risk allele frequency)
\end{itemize}

\end{frame}

\begin{frame}{Technical Details: Example GWAS Summary Statistics Table -- part C}

\begin{table}[h]
\centering
\resizebox{\textwidth}{!}{
\begin{tabular}{|c|c|c|c|c|c|c|c|c|c|c|}
\hline
SNP (rsID) & CHR & BP & A1 & A2 & MAF & BETA & SE & Z-score & P-value & N \\ \hline
rs12345 & 1 & 10583 & A & G & 0.12 & 0.045 & 0.010 & 4.50 & 1.2e-06 & 100000 \\ \hline
rs67890 & 2 & 20345 & C & T & 0.35 & -0.030 & 0.008 & -3.75 & 5.4e-04 & 95000 \\ \hline
rs54321 & 18 & 45678 & G & A & 0.22 & 0.060 & 0.012 & 5.00 & 2.1e-07 & 102000 \\ \hline
\end{tabular}}
\end{table}

\begin{itemize}
\item \textbf{BETA}: Estimated effect size ($\hat{\beta}$), representing the association between the effect allele and the trait. Always check which allele it corresponds to.  
\item \textbf{SE}: Standard error of the effect size estimat, reflecting the uncertainty in the effect size estimate. It is computed as $SE_j = \sqrt{\frac{\hat{\sigma}^2}{\mathbf{X}_j^T \mathbf{X}_j}}$, where $ \hat{\sigma}^2 $ is the estimated residual variance from the model.
\item \textbf{Z}: Standardized test statistic, computed as $Z = \frac{\beta}{\text{SE}}$.  
\item \textbf{P-value}: Significance of the association, testing whether the SNP has an effect on the trait, $p_j = 2 \times (1 - \Phi(|t_j|))$, where $ \Phi $ is the cumulative distribution function (CDF) of the standard normal distribution. Sometimes \texttt{LOG10P} is reported which is the $-\log_{10}P$.
\item \textbf{N}: Number of individuals included in the analysis.  
\item \textbf{N_cases}: Number of individuals with the trait (cases), relevant for case-control studies.  
\item \textbf{N_ctrls}: Number of individuals without the trait (controls), relevant for case-control studies.  
\end{itemize}

\end{frame}

