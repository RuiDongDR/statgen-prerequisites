% Generated slides for Genotype Coding
% Include this file in your main Beamer presentation

\begin{frame}{Section}
\centering
\Huge{Likelihood}
\end{frame}


\begin{frame}{Intuitional Description}
Likelihood is the plausibility of observing your data given a specific model or parameter value.
\end{frame}

\begin{frame}{Graphical Summary}
\includesvg[width=0.8\textwidth]{./cartoons/likelihood.svg}
\end{frame}


\begin{frame}{Key Formula}
The likelihood for a model $\text{M}$ based on data $\text{D}$ is:

$$
\mathcal{L}(\text{M}|\text{D}) = P(\text{D}|\text{M})
$$

Where:
\begin{itemize}
\item (\text{M})$ is the likelihood function for model $\text{M}$ under data $\text{D}$
\item $\text{M}$ represents the model
\item $\text{D}$ represents the observed data
\item $P(\text{D}|\text{M})$ is the probability of observing data $\text{D}$ given the model $\text{M}$
\end{itemize}
\end{frame}


\begin{frame}{Technical Details: Single Sample}
For a single observation $d$:
$\mathcal{L}(d|\text{M}) = P(d|\text{M})$

\end{frame}

\begin{frame}{Technical Details: Multiple Samples (Independence Assumption)}
For multiple independent observations $\text{D} = \{d_1, d_2, ..., d_n\}$:
$$
\mathcal{L}(\text{D}|\text{M}) = \prod_{i=1}^{n} P(d_i|\text{M})
$$

Under the assumption of independence, the joint probability of multiple independent samples is the product of their individual probabilities. 

\end{frame}

\begin{frame}{Technical Details: Log-Likelihood}
We often work with log-likelihood for computational stability:

$\mathcal{L}(\text{D}|\text{M}) = \log \mathcal{L}(\text{D}|\text{M}) = \sum_{i=1}^{n} \log P(d_i|\text{M})$

The log transformation converts products to sums, making calculations easier and preventing numerical underflow with very small probabilities.

In $\log$, unless otherwise specified, we use log base e.

\end{frame}

