% Generated slides for Genotype Coding
% Include this file in your main Beamer presentation

\begin{frame}{Section}
\centering
\Huge{Mediator}
\end{frame}


\begin{frame}{Intuitional Description}
A mediator is a variable that \textbf{sits in the causal pathway between an exposure and an outcome}, explaining the mechanism through which the exposure exerts its effect on the outcome.
\end{frame}

\begin{frame}{Graphical Summary}
\includesvg[width=0.8\textwidth]{./cartoons/mediator.svg}
\end{frame}


\begin{frame}{Key Formula}
The key formula for the concept of a mediator is represented in a causal diagram as:
$$
X → M → Y
$$
Where:
\begin{itemize}
\item $X$ is the independent variable (e.g., genetic variant)
\item $M$ is the mediator variable
\item $Y$ is the dependent variable (e.g., trait)
\item The arrows (→) indicate the direction of causal influence
\end{itemize}

This diagram illustrates that a mediator ($M$) lies in the causal pathway between the independent variable ($X$) and the dependent variable ($Y$). The mediator transmits the effect of $X$ on $Y$, creating an indirect causal pathway through which $X$ affects $Y$.
\end{frame}


\begin{frame}{Technical Details}
A \textbf{mediator} is a variable that explains the mechanism or process by which an independent variable (exposure) influences a dependent variable (outcome). Unlike confounders, mediators are part of the causal pathway and represent how or why an effect occurs.

In formal causal inference terminology, mediation analysis separates the total effect of $X$ on $Y$ into:

1. Direct Effect: The effect of $X$ on $Y$ that does not go through $M$
2. Indirect Effect: The effect of $X$ on $Y$ that operates through $M$

The total effect can be decomposed as:

$$\text{Total Effect} = \text{Direct Effect} + \text{Indirect Effect}$$

Unlike confounders which should be controlled for, adjusting for mediators can block the pathway of interest and mask the total causal effect being studied.
\end{frame}

