% Generated slides for Genotype Coding
% Include this file in your main Beamer presentation

\begin{frame}{Section}
\centering
\Huge{P Value Versus Bayes Factor}
\end{frame}


\begin{frame}{Intuitional Description}
The Bayes factor provides a direct measure of the relative evidence supporting two competing models or hypotheses, while a p-value only measures the compatibility of observed data with a single null hypothesis, making Bayes factors more intuitive for comparing alternative explanations of the data.
\end{frame}

\begin{frame}{Graphical Summary}
\includesvg[width=0.8\textwidth]{./cartoons/p_value_versus_Bayes_factor.svg}
\end{frame}


\begin{frame}{Key Formula}
Under certain conditions, p-values can be converted to minimum Bayes factors (see [Sellke et al., 2021](https://hannig.cloudapps.unc.edu/STOR654/handouts/SellkeBayarriBerger2001.pdf) for more details).
Specifically they show that, provided $p<1/e$, the BF in favor of $H_1$ is not larger than
$$
\text{BF}_\text{min} \approx −e\times p \times \log(p)
$$
\end{frame}


\begin{frame}{Technical Details: p-value}
\begin{itemize}
\item P(data|H0) - probability of data given null hypothesis
\item Measures compatibility of data with null hypothesis
\item Does \textbf{not} directly measure evidence for alternative
\end{itemize}

\end{frame}

\begin{frame}{Technical Details: Bayes Factor}
\begin{itemize}
\item Directly compares evidence for alternative vs. null OR null vs alternative, or any other two models
\item Tells you how much to update your beliefs
\end{itemize}
\end{frame}

