% Generated slides for Genotype Coding
% Include this file in your main Beamer presentation

\begin{frame}{Section}
\centering
\Huge{Collider}
\end{frame}


\begin{frame}{Intuitional Description}
A collider is a variable that is \textbf{influenced by two other variables of interest}, creating a \textbf{spurious} association between them when we condition on (select or control for) the collider in our analysis.
\end{frame}

\begin{frame}{Graphical Summary}
\includesvg[width=0.8\textwidth]{./cartoons/collider.svg}
\end{frame}


\begin{frame}{Key Formula}
The key formula for the concept of a collider is represented in a causal diagram as:
$$
X \rightarrow C \leftarrow Y
$$
Where:
\begin{itemize}
\item $C$ is the collider variable
\item $X$ is one cause of the collider
\item $Y$ is another cause of the collider
\item The arrows (→) indicate the direction of causal influence
\end{itemize}

This diagram illustrates that a collider ($C$) is a variable that is *caused by* both the exposure ($X$) and the outcome ($Y$), creating a situation where $X$ and $Y$ both flow into $C$.

A \textbf{collider} is a variable that is influenced by two other variables in a causal pathway. When we condition on (adjust for, stratify by, or select based on) a collider, we can induce a spurious association between its causes, even if they were originally independent.

\end{frame}


\begin{frame}{Technical Details}
Conditioning on a collider can introduce bias in the estimation of the relationship between its causes. This is often called \textbf{collider bias} or \textbf{selection bias}.

In formal causal inference terminology, conditioning on a collider creates a situation where:
$$
P(Y|X) \neq P(Y|X, C)
$$
Where conditioning on $C$ creates a non-causal association between $X$ and $Y$. This inequality shows that the association between $X$ and $Y$ differs when we condition on the collider.

Key characteristics of collider bias:
1. \textbf{Induced Association}: Conditioning on a collider can create an association between variables that are causally independent
2. \textbf{Berkson's Paradox}: A classic example of collider bias in epidemiology
3. \textbf{Selection Bias}: Often occurs when study participants are selected based on criteria affected by both exposure and outcome
4. \textbf{M-Bias}: A specific pattern in DAGs where conditioning on a collider creates a biased path

Unlike confounders which \textbf{should be controlled} for in analyses, colliders \textbf{should not be controlled} for when estimating the causal effect of interest.
\end{frame}

